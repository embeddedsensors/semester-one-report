%!TEX root = /Users/dylanmorano/Documents/School/Senior/Senior Design/semester-one-report/latexmain/reportone.tex

\chapter{Introduction}

Modern day transportation relies heavily on large scale infrastructures such as: bridges, tunnels and overpasses.  In particular, bridges serve to connect civilizations and provide a vital infrastructure.  Because of this, it is important to maintain a high level of safety and these structures require careful maintenance and monitoring.  The term structural health monitoring (SHM) refers to the process of implementing damage monitoring and a strategy of mitigating damage to engineering structures. 

In the case of suspension bridges, the addition of cables under tension makes monitoring the structure a unique process.  In general, suspension bridges, when properly cared for, have long service lives.  The main cables used to support the hanger cables are composed of bundles, which are wrapped in soft wire and painted for protection.  The wires are also galvanized, and when combined with the wrapping, offer good protection from the elements (Pure Technologies $\#$).  However, as mentioned by Higgens in his report, “a well-documented vulnerability has manifested itself on several bridges in the United States: corrosion of main cables” (Pure Technologies).  Corrosion in the main cables is the primary reason for cables failing and snapping in the main bundle.  When enough cable snaps occur, the loading capabilities of the bridge can be compromised.  Proper monitoring of the bridges and their cables can help to keep the bridge in operation over large periods of time.  However, this is a costly and labor intensive process.

According to a report by Pure Technologies on the Bear mountain Bridge ”A current visual inspection method[s] of a 10ft length of cable, wedging down 5 inches at 8 points around the circumference, only exposes one side of 4000 linear feet of weave, or 0.007$\%$ of the total length” (Pure Technologies $\#$).  It should be noted that this method requires an observer, to visually inspect the wire.  It was found that using this method to expose, inspect, and reinstate the cables costs on the order of \$2 million dollars for 320ft (Pure Technologies $\#$).  

Newer methods for autonomous to semiautonomous monitoring would be beneficial in reducing the cost and labor involved in detecting these cable snaps.  Using sensors to detect when and where cables snaps occurred could provide rapid health assessments of the structure, allowing mitigation steps to be taken.  One possible method is to use acoustic monitoring sensors.  

Acoustic monitoring has the potential to be able to detect breaks in suspension cables.  Breaks in the cables create measurable acoustic events that passive systems can detect.  In addition, active systems can inject signals into the bridge cables, having the sound travel the length of the wire.  This method of monitoring holds the potential to greatly reduce or eliminate the need for cable inspection until a snap occurs.  Furthermore, depending on package design, acoustic sensing could greatly reduce cost of operations.


\section{Objectives}

\subsection{Phase one (Fall 2014)}

The objective of the first phase of this study was to explore a methodology for determining the best frequencies of propagation on a suspension bridge main cable. A method was explored using the excitation of small-scale steel rods in order to be applied to large scale system. This was to involve the testing of theoretical calculations to determine whether or not resonant frequencies could be accurately predicted using the methods described in section (SITE THEORY SECTION HERE). Experiments were to be carried out on a small scale steel rod for several types of excitations. The results of which were to be compared to the expected theoretical calculation to verify the accuracy of the prediction methods. Resonant frequencies were selected to be the best frequencies of propagation due to the fact that they cary the best in a medium. Injection of a signal at the resonant frequency of a medium would require less amplitude than other spectra and might result in a stronger response of returned signal. 

